{
date:"2012-2-11 8:27",
title:"About ergo-cms",
metadesc: "An introduction on ergo-cms",
metakeys: "about",
extracss: "aboutpage",
}

Ergo was written, mainly to scratch an *itch*. I had been waiting for several years for Jekyll, Hyde, Ghost, etc, etc to make some realisations:

* Not everybody likes markdown!
* Where is a nice gui to do stuff?

This project originally started as a C++ static website generator. It used markdown... but then I found "Textile":http://txstyle.org and then ported it to @node.js@ as quickly as my fingers could type. I think it took a few hours. At one point I gave some notion to refactoring a lot of the code and got 99% of the way, but it wasn't production ready. That code exists still, but is horrible and clunky and not extensible. 
This project is the 4th generation of something that no-one else other than myself has seen.


This is what ergo-cms is:

* A *very* lightweight cms. I was shocked to discover than an install of Ghost currently clocks in at around 185MB! Ergo doesn't use *anyone* elses fancy modules, unless it's vital. Hence there's no mustache, less, or any of that stuff here. 
* Very extensible. If you like you *can* use mustache, or less, etc.
* Database-less. It uses flat files. It natively supports Markdown (for those who haven't seen the light), and Textile (for those that have). Being very extensible, it can support a whatever you can dream up.
* Great at creating static websites. 
* Great at publishing minimal changes to your webserver (using rsync)


What ergo-cms isn't:

* A Jekyll, WordPress, or <insert your cms name here> replacement.
* A web server. Although there's the ability to view your files in a locally hosted server (http:localhost:8181), it's nothing more than a simple file server & not production hardened. There's no reason for such a thing, with a flat file generator.
* Supportive of ruby-esque things. YAML is one of the most crazy things I've ever seen. But people like coffee script too. If you want to write a plugin, by all means, it'll work nicely here - but I won't write it. I'm not a fan of fan-boi-itis. By supporting markdown, I believe supported enough of all of _that_ kind of craziness, & won't go a step further!

What it will (eventually) do:

# Have two admin interfaces: one running on php that 'prods' a version of ergo-cli, and another that uses express (or something)
# Have an offline editor. This will be a 'distraction free' app, with very minimal toys. It will probably run with electron and be _beautiful_ to behold (because sometimes you need to be inspired when writing, I've found).


So, there you have it. Ergo is yet another flat file cms static website generator, but is easy to use and has plans to have a proper admin interface and offline editor.

