title = JSINF v1.0 Spec [draft]
metadesc = Specifications for JSINF, a human readable configuration, v1
###

Here's a table of what I suggest for the JSINF, and what I'll be using for "ergo-cms":https://ergo-cms.github.io. This spec will be updated here (or links to updated spec). The javascript implementation shall be the final arbitrar of ambiguity (unless there's a clear fault in it).

A Javascript implementation and this spec will be available at "https://github.com/cmroanirgo/jsinf":https://github.com/cmroanirgo/jsinf.

h2. Specification [draft]

In all cases @\n@ should be treated as @\r?\n@, to be line-endings agnostic.

The use-case of this specification is to produce a hierarchical data structure similar to that of JSON, or more specificially a Javascript object. Note however, that the format should be able to be searched for a value, just like an .INI file, without needing to decode and load the whole file.

Errors are to be expected in the file. They should _*never*_ generate an assert in production code, but is acceptable behaviour in test cases only. So, a prime use-case is to feed the deocder absolute junk (eg an image file). It will return an object, with nothing in it. This is an _error last_ approach. It is up to the rest of the code to decide that an empty object is acceptable or not.

h3. Natural Names & Section Names

We define a @natural-name@ as a regular expression: @[\w\-\.\/]*[\w]@. This a normal word (a-zA-Z0-9 & _) & '-', '/' & '.', but generally not ending with the symbols (except _). These are valid:

* SomeKey
* Some__--..//key2
* ../some/key_

These are NOT valid:

* key/../
* #key
* key-
* some key

A @section-name@ is the same as a @natural-name@ except that spaces are allowed when used in the @[section name]@ blocks. See below.

h3. Embeded Code

The specification allows for code to be entered. The actual language of this code is up to the implementation, so a Perl implementation may render the code blocks as Perl. The Javascript implementation will render as Javascript and return a Javascript Object.

h3. The Rest

Where @section@ is: @/\n\s*\[(natural-name)\]\s*(?:\n)/@. @$1@ contains the section name. The one caveat is that a section name in @[ ]@ can also contain spaces.

Where @comment@ is: @/\n[#;].*?@

Where @key@ suits the regular expression: @/\n\s*(natural-name)\s*(?:\=\s*)/@. That is, a natural-word start at or near the start of a new line, with an equals sign after it. There need NOT be a value to the right of the equals. @$1@ contains the key

Where @line-starters@ is either a comment, key, or section: @/(?:comment|key|section)/@

Where @code-value@ suits the regular expression: @/\=\s*(\{ (.*)?\})\s*$/@. A code value continues to the end of a line. This may need ironing out a little more, to weigh security vs usability. This should always ONLY be a single line match, to encourage simplicity. @$1@ contains the 'code'.

Similarly, @value@ suits the regular expression: @/\=\s*([\s\S]*?)(?:line-starters|$)/@. Matches all text until the next 'key =', [section], #comment, or end. @$1@ contains the value.

Where @default-value@ is: @/\s*([\s\S])*\s*(?:line-starters|$)/@. Similar to @value@, but without need for the @\=\s*@ at the start. @$1@ contains the value.

A @escaped-value-block@ suits the regular expression: @/-{3,}\s*\n([\s\S])*\s*\n(?:-{3,}|$)\s*/@. A block with --- and --- surrounding it. @$1@ contains the value.



<pre style="min-width:860px"><code>title = Welcome to my homepage                               ; key + value
uri = index.html                                                        ; key + value
date={   (new Date()).toString + " OK"; }                               ; key + code+value
metakeys = Some keys,other keys                                         ; key + value
metadesc =                                                              ; key + ...
I like writing blogs and learning                                       ;     ... continues
about myself. Trailing spaces are                                       ;     ... continue
eaten                                                                   ;     ... value

[section1]                                                              ; section
content = This is the content for some magical 'section 1' that exists. ; key + ...
It keeps going until another '[section]' is found (on it's own line),   ;     ... continues
or a 'key=' is met. Having a newline above [section] is optional.       ;     ... value
extracss = green                                                        ; key + value

#This is a comment. It needs to start with a "#" or                     ; comment
; a ";"                                                                 ; comment

section2.content = Another sections, using 'dot' notation.              ; key + ...
These are also valid: [section.sub.subsub], section.sub.subsub.key =    ;      ... value

[main content]                                                          ; section
extracss = orange                                                       ; key + value

Each section can also have it's own text. It must have a blank line     ; default-value 
above it, OR be directly beneath a [section] mark                       ; ...(for main.content)
And keeps on going until a valid [section], #comment, [section] or key=
is reached.

[footer content]
-------------                                        ; escaped-value-block, for 'footer content' default
#This is footer content. This is NOT a comment
;This is NOT a comment
not-a-key = NOT a value

All of this text is a 'default-value-block' (for footer.content). It is OK
to forget the 3 dashes below, if the end of the file is reached.
---         ; the end of the block

[final.footer]
some.key =                                           ; key (+ an empty value)
---                                                  ; escaped-value-block, for 'final.footer' default
This is the last block of text. 
This is valid.
</code></pre>






